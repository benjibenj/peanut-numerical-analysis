\documentclass[10pt]{article}
\usepackage{amsmath}
\usepackage{latexsym}
\usepackage{amsfonts}
\usepackage[normalem]{ulem}
\usepackage{soul}
\usepackage{array}
\usepackage{amssymb}
\usepackage{extarrows}
\usepackage{graphicx}
\usepackage[backend=biber,
style=numeric,
sorting=none,
isbn=false,
doi=false,
url=false,
]{biblatex}\addbibresource{bibliography.bib}

\usepackage{subfig}
\usepackage{wrapfig}
\usepackage{wasysym}
\usepackage{enumitem}
\usepackage{adjustbox}
\usepackage{ragged2e}
\usepackage[svgnames,table]{xcolor}
\usepackage{tikz}
\usepackage{longtable}
\usepackage{changepage}
\usepackage{setspace}
\usepackage{hhline}
\usepackage{multicol}
\usepackage{tabto}
\usepackage{float}
\usepackage{multirow}
\usepackage{makecell}
\usepackage{fancyhdr}
\usepackage[toc,page]{appendix}
\usepackage[hidelinks]{hyperref}
\usetikzlibrary{shapes.symbols,shapes.geometric,shadows,arrows.meta}
\tikzset{>={Latex[width=1.5mm,length=2mm]}}
\usepackage{flowchart}\usepackage[paperheight=11.0in,paperwidth=8.5in,left=1.18in,right=1.18in,top=0.98in,bottom=0.98in,headheight=1in]{geometry}
\usepackage[utf8]{inputenc}
\usepackage[T1]{fontenc}
\TabPositions{0.5in,1.0in,1.5in,2.0in,2.5in,3.0in,3.5in,4.0in,4.5in,5.0in,5.5in,6.0in,}

\urlstyle{same}

\renewcommand{\_}{\kern-1.5pt\textunderscore\kern-1.5pt}

 %%%%%%%%%%%%  Set Depths for Sections  %%%%%%%%%%%%%%

% 1) Section
% 1.1) SubSection
% 1.1.1) SubSubSection
% 1.1.1.1) Paragraph
% 1.1.1.1.1) Subparagraph


\setcounter{tocdepth}{5}
\setcounter{secnumdepth}{5}


 %%%%%%%%%%%%  Set Depths for Nested Lists created by \begin{enumerate}  %%%%%%%%%%%%%%


\setlistdepth{9}
\renewlist{enumerate}{enumerate}{9}
		\setlist[enumerate,1]{label=\arabic*)}
		\setlist[enumerate,2]{label=\alph*)}
		\setlist[enumerate,3]{label=(\roman*)}
		\setlist[enumerate,4]{label=(\arabic*)}
		\setlist[enumerate,5]{label=(\Alph*)}
		\setlist[enumerate,6]{label=(\Roman*)}
		\setlist[enumerate,7]{label=\arabic*}
		\setlist[enumerate,8]{label=\alph*}
		\setlist[enumerate,9]{label=\roman*}

\renewlist{itemize}{itemize}{9}
		\setlist[itemize]{label=$\cdot$}
		\setlist[itemize,1]{label=\textbullet}
		\setlist[itemize,2]{label=$\circ$}
		\setlist[itemize,3]{label=$\ast$}
		\setlist[itemize,4]{label=$\dagger$}
		\setlist[itemize,5]{label=$\triangleright$}
		\setlist[itemize,6]{label=$\bigstar$}
		\setlist[itemize,7]{label=$\blacklozenge$}
		\setlist[itemize,8]{label=$\prime$}

\setlength{\topsep}{0pt}\setlength{\parskip}{8.04pt}
\setlength{\parindent}{0pt}

 %%%%%%%%%%%%  This sets linespacing (verticle gap between Lines) Default=1 %%%%%%%%%%%%%%


\renewcommand{\arraystretch}{1.3}


%%%%%%%%%%%%%%%%%%%% Document code starts here %%%%%%%%%%%%%%%%%%%%



\begin{document}
\textbf{\uline{False Position }}\par

$\ast$ The user will need to be sure that the function is continuous on the interval. \par

\begin{enumerate}
	\item Ask the user for a function, the tolerance, and the max number of iterations\par

	\item Ask the user for the values A\textbf{ }and B, which will be the values of the initial interval.\par

	\item We create a variable i to count the number of iterations. We begin with 1. \par

	\item We evaluate A\textbf{ }and B in the function to obtain f(A) and f(B). If f(A)$\ast$ f(B) = 0, tell the user that this is the root\par

	\item We do a conditional to begin the execution of the method: if f(A)$\ast$ f(B) $ \leq $  0 then execute$ \ldots $  to verify that in the interval there is a root. In the case that there is not then tell the user that in this interval there is no root.\par

	\item Error = $ \vert $ A-B $ \vert $ \par

	\item We find the middle value of the interval. M = A-(f(A)$\ast$ (A-B))/ (f(A)-f(B)) and evaluate in the function to obtain f(M)\par

	\item Cycle: while the error > tolerance, i < max number of iterations, f(M) $ \neq $  0, f(A)$\ast$ f(B) < 0, do:\par

\begin{enumerate}
	\item If f(A) $\ast$  f(M) < 0\par

\begin{enumerate}
	\item B = M \par

	\item  f(B) = f(M)\par

	\item M = A –(f(A)$\ast$ (A-B))/(f(A) – f(B))\par

	\item f(M) = evaluate the new M in the function\par

	\item i = i +1\par

	\item Error = $ \vert $ A-B $ \vert $ \par


\end{enumerate}
	\item If f(B) $\ast$  f(M) < 0\par

\begin{enumerate}
	\item A = M \par

	\item f(A) = f(M)\par

	\item M = A-(f(A)$\ast$ (A-B)) / (f(A)-f(B))\par

	\item f(M) = evaluate the new M in the function \par

	\item i = i +1\par

	\item Error = $ \vert $ A-B $ \vert $ \par


\end{enumerate}
\end{enumerate}
	\item If the error $ \leq $  tolerance, tell the user that the root is in the interval [A,B]\ ( with the final value) with an error of : \_\_\_ (with the final value of the error)\par

	\item If f(M) = 0 tell the use that M is the root.\par

	\item If i\ = max number of iterations, tell the user that he has reached the limit of the given iterations and the root is in the interval [A,B] ( with the final values) with an error of: \_\_\_\_ (with the final error value)
\end{enumerate}\par


\vspace{\baselineskip}

\printbibliography
\end{document}