\documentclass[12pt]{article}
\usepackage{amsmath}
\usepackage{latexsym}
\usepackage{amsfonts}
\usepackage[normalem]{ulem}
\usepackage{soul}
\usepackage{array}
\usepackage{amssymb}
\usepackage{extarrows}
\usepackage{graphicx}
\usepackage[backend=biber,
style=numeric,
sorting=none,
isbn=false,
doi=false,
url=false,
]{biblatex}\addbibresource{bibliography.bib}

\usepackage{subfig}
\usepackage{wrapfig}
\usepackage{wasysym}
\usepackage{enumitem}
\usepackage{adjustbox}
\usepackage{ragged2e}
\usepackage[svgnames,table]{xcolor}
\usepackage{tikz}
\usepackage{longtable}
\usepackage{changepage}
\usepackage{setspace}
\usepackage{hhline}
\usepackage{multicol}
\usepackage{tabto}
\usepackage{float}
\usepackage{multirow}
\usepackage{makecell}
\usepackage{fancyhdr}
\usepackage[toc,page]{appendix}
\usepackage[hidelinks]{hyperref}
\usetikzlibrary{shapes.symbols,shapes.geometric,shadows,arrows.meta}
\tikzset{>={Latex[width=1.5mm,length=2mm]}}
\usepackage{flowchart}\usepackage[paperheight=11.0in,paperwidth=8.5in,left=1.18in,right=1.18in,top=0.98in,bottom=0.98in,headheight=1in]{geometry}
\usepackage[utf8]{inputenc}
\usepackage[T1]{fontenc}
\TabPositions{0.5in,1.0in,1.5in,2.0in,2.5in,3.0in,3.5in,4.0in,4.5in,5.0in,5.5in,6.0in,}

\urlstyle{same}

\renewcommand{\_}{\kern-1.5pt\textunderscore\kern-1.5pt}

 %%%%%%%%%%%%  Set Depths for Sections  %%%%%%%%%%%%%%

% 1) Section
% 1.1) SubSection
% 1.1.1) SubSubSection
% 1.1.1.1) Paragraph
% 1.1.1.1.1) Subparagraph


\setcounter{tocdepth}{5}
\setcounter{secnumdepth}{5}


 %%%%%%%%%%%%  Set Depths for Nested Lists created by \begin{enumerate}  %%%%%%%%%%%%%%


\setlistdepth{9}
\renewlist{enumerate}{enumerate}{9}
		\setlist[enumerate,1]{label=\arabic*)}
		\setlist[enumerate,2]{label=\alph*)}
		\setlist[enumerate,3]{label=(\roman*)}
		\setlist[enumerate,4]{label=(\arabic*)}
		\setlist[enumerate,5]{label=(\alph*)}
		\setlist[enumerate,6]{label=(\Roman*)}
		\setlist[enumerate,7]{label=\arabic*}
		\setlist[enumerate,8]{label=\alph*}
		\setlist[enumerate,9]{label=\roman*}

\renewlist{itemize}{itemize}{9}
		\setlist[itemize]{label=$\cdot$}
		\setlist[itemize,1]{label=\textbullet}
		\setlist[itemize,2]{label=$\circ$}
		\setlist[itemize,3]{label=$\ast$}
		\setlist[itemize,4]{label=$\dagger$}
		\setlist[itemize,5]{label=$\triangleright$}
		\setlist[itemize,6]{label=$\bigstar$}
		\setlist[itemize,7]{label=$\blacklozenge$}
		\setlist[itemize,8]{label=$\prime$}

\setlength{\topsep}{0pt}\setlength{\parskip}{8.04pt}
\setlength{\parindent}{0pt}

 %%%%%%%%%%%%  This sets linespacing (verticle gap between Lines) Default=1 %%%%%%%%%%%%%%


\renewcommand{\arraystretch}{1.3}


%%%%%%%%%%%%%%%%%%%% Document code starts here %%%%%%%%%%%%%%%%%%%%



\begin{document}
\textbf{\uline{Gaussian Elimination with Total Pivoting}}\par


\vspace{\baselineskip}
\begin{enumerate}
	\item Ask the user for matrix A \par

	\item Ask the user for a vector b and whole number n\par

\begin{enumerate}
	\item For k = 1 until n-1; with step 1\par

\begin{enumerate}
	\item For p = k until n; with step 1\par

	\item For r = k until n; with step 1\par

	\item a\textsubscript{kk = }max $ \vert $ a\textsubscript{pr}$ \vert $ \par


\end{enumerate}
	\item If a\textsubscript{kk} = 0 \par

\begin{enumerate}
	\item Print $``$Use another method$"$ \par


\end{enumerate}
	\item If not \par

\begin{enumerate}
	\item If k $ \neq $  p and k $ \neq $  r\par

	\item Change row k for p\par

	\item Change column k for r\par

	\item End of if\par


\end{enumerate}
	\item End of if\par

	\item For i = k +1 until n; with step 1 \par

\begin{enumerate}
	\item m\textsubscript{ik }= a\textsubscript{ij/}a\textsubscript{kk}\par

	\item if m\textsubscript{ik} < 1\par

\begin{enumerate}
	\item For j = k until n+1; with step 1 \par

	\item a\textsubscript{ij = }a\textsubscript{ij }– m\textsubscript{ik $\ast$ } a\textsubscript{ik}\par


\end{enumerate}
	\item End of if\par


\end{enumerate}
	\item x = Progressive Replacement(ub,n)\par

\begin{enumerate}
	\item for i = n < 1; with 1 step\par

	\item sum = 0\par

\begin{enumerate}
	\item for j = i + 1 < 1; with 1 step\par

\begin{enumerate}
	\item sum = sum + a\textsubscript{ij }$\ast$  x\textsubscript{ij}\par


\end{enumerate}
	\item x\textsubscript{i }= (a\textsubscript{in }+ 1 – sum)/ a\textsubscript{ij}
\end{enumerate}
\end{enumerate}
\end{enumerate}
\end{enumerate}\par


\vspace{\baselineskip}

\printbibliography
\end{document}