\documentclass[12pt]{article}
\usepackage{amsmath}
\usepackage{latexsym}
\usepackage{amsfonts}
\usepackage[normalem]{ulem}
\usepackage{soul}
\usepackage{array}
\usepackage{amssymb}
\usepackage{extarrows}
\usepackage{graphicx}
\usepackage[backend=biber,
style=numeric,
sorting=none,
isbn=false,
doi=false,
url=false,
]{biblatex}\addbibresource{bibliography.bib}

\usepackage{subfig}
\usepackage{wrapfig}
\usepackage{wasysym}
\usepackage{enumitem}
\usepackage{adjustbox}
\usepackage{ragged2e}
\usepackage[svgnames,table]{xcolor}
\usepackage{tikz}
\usepackage{longtable}
\usepackage{changepage}
\usepackage{setspace}
\usepackage{hhline}
\usepackage{multicol}
\usepackage{tabto}
\usepackage{float}
\usepackage{multirow}
\usepackage{makecell}
\usepackage{fancyhdr}
\usepackage[toc,page]{appendix}
\usepackage[hidelinks]{hyperref}
\usetikzlibrary{shapes.symbols,shapes.geometric,shadows,arrows.meta}
\tikzset{>={Latex[width=1.5mm,length=2mm]}}
\usepackage{flowchart}\usepackage[paperheight=11.0in,paperwidth=8.5in,left=1.18in,right=1.18in,top=0.98in,bottom=0.98in,headheight=1in]{geometry}
\usepackage[utf8]{inputenc}
\usepackage[T1]{fontenc}
\TabPositions{0.5in,1.0in,1.5in,2.0in,2.5in,3.0in,3.5in,4.0in,4.5in,5.0in,5.5in,6.0in,}

\urlstyle{same}

\renewcommand{\_}{\kern-1.5pt\textunderscore\kern-1.5pt}

 %%%%%%%%%%%%  Set Depths for Sections  %%%%%%%%%%%%%%

% 1) Section
% 1.1) SubSection
% 1.1.1) SubSubSection
% 1.1.1.1) Paragraph
% 1.1.1.1.1) Subparagraph


\setcounter{tocdepth}{5}
\setcounter{secnumdepth}{5}


 %%%%%%%%%%%%  Set Depths for Nested Lists created by \begin{enumerate}  %%%%%%%%%%%%%%


\setlistdepth{9}
\renewlist{enumerate}{enumerate}{9}
		\setlist[enumerate,1]{label=\arabic*)}
		\setlist[enumerate,2]{label=\alph*)}
		\setlist[enumerate,3]{label=(\roman*)}
		\setlist[enumerate,4]{label=(\arabic*)}
		\setlist[enumerate,5]{label=(\Alph*)}
		\setlist[enumerate,6]{label=(\Roman*)}
		\setlist[enumerate,7]{label=\arabic*}
		\setlist[enumerate,8]{label=\alph*}
		\setlist[enumerate,9]{label=\roman*}

\renewlist{itemize}{itemize}{9}
		\setlist[itemize]{label=$\cdot$}
		\setlist[itemize,1]{label=\textbullet}
		\setlist[itemize,2]{label=$\circ$}
		\setlist[itemize,3]{label=$\ast$}
		\setlist[itemize,4]{label=$\dagger$}
		\setlist[itemize,5]{label=$\triangleright$}
		\setlist[itemize,6]{label=$\bigstar$}
		\setlist[itemize,7]{label=$\blacklozenge$}
		\setlist[itemize,8]{label=$\prime$}

\setlength{\topsep}{0pt}\setlength{\parskip}{8.04pt}
\setlength{\parindent}{0pt}

 %%%%%%%%%%%%  This sets linespacing (verticle gap between Lines) Default=1 %%%%%%%%%%%%%%


\renewcommand{\arraystretch}{1.3}


%%%%%%%%%%%%%%%%%%%% Document code starts here %%%%%%%%%%%%%%%%%%%%



\begin{document}
\textbf{\uline{Jacobi, Gauss Seidel y Gauss Seidel (SOR)}}\par


\vspace{\baselineskip}
\begin{enumerate}
	\item We ask the user for a matrix. We call the matrix A. The matrix must be a square matrix.\par

	\item We ask for a vector b and we make sure that it is the same length as matrix A.\par

	\item Now ask the user for an initial value approximation which we will call x0, a tolerance (tol) and we make sure it is not negative, we also aks for a maximum number of iterations(Nmax), which also has to be nonnegative, l (it will be 1 if you want to execute Jacobi, or 2 if you want to execute Guass Seidel) y w if you want to use the method of relation. \par

	\item We make sure that A does not have any 0’s in the diagonal, and that the det(A) $ \neq $  0. \par

	\item Now make 3 new matrices, named D, L and U. All must be the same size as A and must have the following parameters:\par

\begin{enumerate}
	\item  D must be a matrix that has the same main diagonal elements that matrix A has and the rest of the elements must be 0.\par

	\item  L will have the same elements that are below the diagonal in A but with the opposite sign. The diagonal in L will be the same as A, and all other elements will be 0. \par

	\item U will have the same elements in A that are above the diagonal with opposite signs, and will have the same elements in the diagonal as A and the rest of the elements will be 0.\par


\end{enumerate}
	\item Now we begin to execute the method. \par

\begin{enumerate}
	\item If l is equal to 1, execute Jacobi method,\par

\begin{enumerate}
	\item T = D\textsuperscript{-1 }$\ast$  (L+U)\par

	\item C = D\textsuperscript{-1\  }$\ast$  b\par


\end{enumerate}
	\item If l is equal to 2 execute the Gauss Seidel Method: \par

\begin{enumerate}
	\item T = (D-L)\textsuperscript{-1 }$\ast$  U\par

	\item C = (D-L)\textsuperscript{-1 }$\ast$  b\par


\end{enumerate}
	\item All other cases execute SOR\par

\begin{enumerate}
	\item T = (D-w$\ast$ L)\textsuperscript{-1 }$\ast$  ((1-w)$\ast$ (D+w$\ast$ U))\par

	\item C = (w$\ast$ (D-w$\ast$ L))\textsuperscript{-1} $\ast$  b\par


\end{enumerate}
\end{enumerate}
	\item Now we find and make the spectral radiance as the as the absolute value of the largest proper values making sure they are less than 1. \par

	\item Now we make a counter for the iterations beginning with 0, a variable for the error which begins with 1, and a variable for the old X , we will call Xant, and we will start it off as Xant=x0 our initial approximation\par

	\item Now we make a cycle, while the Error > the tolerance , iterations(counter) < Nmax, Do:\par

\begin{enumerate}
	\item Xact = Xant$\ast$ T+C\par

	\item Error = norm of Xant – Xact\par

	\item Xant = Xact\par

	\item Counter++ 
\end{enumerate}
\end{enumerate}\par


\vspace{\baselineskip}

\printbibliography
\end{document}